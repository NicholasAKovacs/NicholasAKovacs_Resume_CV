%!TEX TS-program = xelatex
%!TEX encoding = UTF-8 Unicode
% Awesome CV LaTeX Template for CV/Resume
%
% This template has been downloaded from:
% https://github.com/posquit0/Awesome-CV
%
% Author:
% Claud D. Park <posquit0.bj@gmail.com>
% http://www.posquit0.com
%
% Template license:
% CC BY-SA 4.0 (https://creativecommons.org/licenses/by-sa/4.0/)
%


%-------------------------------------------------------------------------------
% CONFIGURATIONS
%-------------------------------------------------------------------------------
% A4 paper size by default, use 'letterpaper' for US letter
\documentclass[11pt, letterpaper]{CV_latex_class}

% Configure page margins with geometry
\geometry{left=1.4cm, top=1.4cm, right=1.4cm, bottom=1.4cm, footskip=.5cm}

% Specify the location of the included fonts
\fontdir[fonts/]

% Color for highlights
% Awesome Colors: awesome-emerald, awesome-skyblue, awesome-red, awesome-pink, awesome-orange
%                 awesome-nephritis, awesome-concrete, awesome-darknight
\colorlet{awesome}{awesome-red}
% Uncomment if you would like to specify your own color
% \definecolor{awesome}{HTML}{CA63A8}

% Colors for text
% Uncomment if you would like to specify your own color
% \definecolor{darktext}{HTML}{414141}
% \definecolor{text}{HTML}{333333}
% \definecolor{graytext}{HTML}{5D5D5D}
% \definecolor{lighttext}{HTML}{999999}

% Set false if you don't want to highlight section with awesome color
\setbool{acvSectionColorHighlight}{true}

% If you would like to change the social information separator from a pipe (|) to something else
\renewcommand{\acvHeaderSocialSep}{\quad\textbar\quad}


%-------------------------------------------------------------------------------
%	PERSONAL INFORMATION
%	Comment any of the lines below if they are not required
%-------------------------------------------------------------------------------
% Available options: circle|rectangle,edge/noedge,left/right
% \photo{./examples/profile.png}
\name{}{Nicholas Attila Kovacs}
\position{Bioinformaticist \& Data Scientist}
%\address{Dual US and Hungarian Citizen}

\mobile{(248) 895-2704}
\email{NAttilaKovacs@gmail.com}
%\citizenship{US Citizen}
\homepage{NicholasAKovacs.com}
\github{NicholasAKovacs}
\linkedin{NicholasAKovacs}
% \gitlab{gitlab-id}
% \stackoverflow{SO-id}{SO-name}
% \twitter{@twit}
% \skype{skype-id}
% \reddit{reddit-id}
% \extrainfo{extra informations}

%\quote{``Be the change that you want to see in the world."}

%-------------------------------------------------------------------------------
%	LETTER INFORMATION
%	All of the below lines must be filled out
%-------------------------------------------------------------------------------
% The company being applied to
%\recipient
 % {}
 % {}
% The date on the letter, default is the date of compilation
%\letterdate{\today}
% The title of the letter
\lettertitle{Application for POSTDOCTORAL FELLOW POPULATION ANALYTICS, Requisition ID: 6424180319}
% How the letter is opened
\letteropening{Dear Hiring Staff at the Computational Science group in the Discovery Sciences organization,}
% How the letter is closed
\letterclosing{Sincerely,}
% Any enclosures with the letter
%\letterenclosure[See attached]{Resume}

%-------------------------------------------------------------------------------
\begin{document}

% Print the header with above personal informations
% Give optional argument to change alignment(C: center, L: left, R: right)
\makecvheader

% Print the footer with 3 arguments(<left>, <center>, <right>)
% Leave any of these blank if they are not needed
\makecvfooter
  %{\today}
  {Nicholas Attila Kovacs~~~·~~~US Citizen}
  %{\thepage}

% Print the title with above letter informations
\makelettertitle

%-------------------------------------------------------------------------------
%	LETTER CONTENT
%-------------------------------------------------------------------------------
\begin{cvletter}

\lettersection{About Me}

\hspace{1em} Transitioning from an experimental, bench-top biochemist in undergrad to a computational biologist in graduate school is a transition I am very proud of because I became the go-to guy in my lab and department to analyze data which led to my involvement in a diverse range of projects spanning drug development, nanoparticles, and the origin and diversification of life. I chose my PhD project because I wanted to work on answering the question that inspired me to pursue a career in science, ``how did life originate and evolve?", but I realized during my PhD that I am much more interested in applied science. My changing interests are apparent in 2 of my coauthored articles concerning the development of anti-fungal therapeutics and in my 3rd, 1st-author publication of my PhD where I compare more than 400 ribosomal structures that are available on the Protein Data Bank with a focus on the structures of ribosomes from disease-causing microbes \textit{Leishmania donovani}, \textit{Plasmodium falciparum}, and \textit{Trypansosoma brucei} and how these differences can lead to the development of new therapeutics.

\hspace{1em}My favorite courses I have taken are ``Programming for Bioinformatics" and ``Computational Genomics" which are taught by my PhD thesis committee member and letter of reference writer, Dr. King Jordan. In programming for bioinformatics, I completed assignments in bash and perl that taught me good practices and familiarized me with basic bioinformatics software such as Bioperl and BLAST. In computational genomics, I worked in teams to to assemble genomes and identify the species and strains from which they were isolated from via Hi-Seq NGS reads provided by the CDC. I also really enjoyed two computer science graduate courses which I completed in my preferred programming language, Python; in Computing for Bioinformatics, I learned data structures and how to design efficient algorithms, and in Network Science I identified network motifs in protein-protein interaction data and did a pandemic analysis on epidemiology data. I was awarded an ``A" in all 4 of these courses.  

\lettersection{Why the Postdoctoral Fellow Population Analytics?}

\hspace{1em} The Janssen Postdoctoral Program is the perfect program for me to apply my skills and expertise gained in graduate school to improve the well-being of others. The Postdoctoral Fellow Population Analytics position will allow me to apply my skills in computational biology and data science to longitudinal, prospective, population-based studies of genetic, environment, and lifestyle data so that I can generate new knowledge that can be used to intercept disease. I will get to collaborate with many different experimental and computational scientists to accomplish the goals set in front of me and present my findings to collaborators within and outside Janssen. I hope that I am offered a position at Janssen as the principal investigator of the projects I work on after this 2 year postdoc.


\lettersection{Why Me?}

\hspace{1em} I am a productive and interdisciplinary scientist that works well in multidisciplinary teams and has a strong desire to have a direct impact on the well-being of others. I like to stay up to date with new technologies, adhere to reproducibility and version control, and love teaching others. Having recently been awarded a PhD in bioinformatics with a minor in biochemistry, I am eager to apply my expertise to new problems, while improving my foundational skills and learning new methodologies. The position of Postdoctoral Fellow Population Analytics at Janssen fits very well with what I want to to do as an early-career scientist and my skills and interests are more than sufficient for the required and preferred qualifications. My desired start date is approximately February 1, 2019.

\end{cvletter}


%-------------------------------------------------------------------------------
% Print the signature and enclosures with above letter informations
\makeletterclosing

%-------------------------------------------------------------------------------
\end{document}
