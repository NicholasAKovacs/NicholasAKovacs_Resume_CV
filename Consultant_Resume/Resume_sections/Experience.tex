%-------------------------------------------------------------------------------
%	SECTION TITLE
%-------------------------------------------------------------------------------
\cvsection{Experience}


%-------------------------------------------------------------------------------
%	CONTENT
%-------------------------------------------------------------------------------
\begin{cventries}

%---------------------------------------------------------
  \cventry
    {Georgia Institute of Technology} % Job title
    {\href{https://scholar.google.com/citations?hl=en&user=KIFFrX4AAAAJ&view_op=list_works&gmla=AJsN-F6ZS9ZeLqMG4o6JCZS87HLjoP76wA4kFValw4xx0A6ZWJJoxCWe2xu6VZvt_LgEw9YpCglVTlKY_5PfKpxiyghYFxLPNm5zrUwdBE5Fn-Sr098nt7NJGM2cH5l47IrTsSra8Vob}{Graduate Research Assistant}} % Organization
    {Atlanta, GA} % Location
    {Aug 2013 - Current} % Date(s)
    {{\textbf{Dissertation}: \textit{Data Mining the Atomic Structure of the Ribosome to Unravel the History of Protein Folding}}
      \begin{cvitems} % Description(s) of tasks/responsibilities
      \vspace{4.0mm}
        \item \textbf{Summary}: Applied structural bioinformatics methods and machine learning algorithms such as clustering to atomic coordinate datasets from over 100 biomolecules composed of 150,000-200,000 atoms to unravel the interralatedness and origin of life.
        \item \textbf{Results}: Two 1\textsuperscript{st}-author research articles published. 3\textsuperscript{rd} in preparation.
        \item \textbf{Funding}: NASA Astrobiology Institute
        \item \textbf{Collaboration}: Provided computational analysis for coworkers projects; resulted in coauthor of 2 experimental and 4 computational research articles.
        \item \textbf{Communication}: Independently wrote and awarded \$7,000 NSF grant to support summer research in Taiwan. Oral and poster presentations at 7 domestic and international scientific conferences.
        \item \textbf{Mentoring}: Awarded \$2,500 conference and travel funding for mentoring undergraduate student.
        \item \textbf{Courses}: 9 courses in biochemisty, computational biology, statistics, and computer science. Concept-to-Market business short-course completed.
      \end{cvitems}
    }

 %---------------------------------------------------------
  \cventry
    {Course Project for CS 7280 - Network Science} % Job title
    {\href{https://nbviewer.jupyter.org/github/NicholasAKovacs/CS7280_Ribosomal_Network_Analysis/blob/master/Jupyter_Notebooks/EsCo_complete_and_SaCe_rPro_analysis/Interaction_Network_Analysis_of_Complete_Ecoli_ribosome_and_Yeast_LSU_rProteins.ipynb}{Atomic Interaction Network Analysis of the Ribosome}} % Organization
    {Atlanta, GA} % Location
    {Fall 2017} % Date(s)
    {
      \begin{cvitems} % Description(s) of tasks/responsibilities
      	\item Collaborated with a team member to apply course concepts and algorithms to 3 atomic interaction networks of the biomolecule, the ribosome, each composed of more than 100,000 edges between approx. 50,000 nodes. 
      	\item {\bf Results}: Predicted RNA and protein folding domains within the ribosome by applying community detection algorithms.
      \end{cvitems}
    }

 %---------------------------------------------------------
  \cventry
    {Georgia Institute of Technology} % Job title
    {Graduate Teaching Assistant} % Organization
    {Atlanta, GA} % Location
    {Aug 2013 - Dec 2016} % Date(s)
    {
      \begin{cvitems} % Description(s) of tasks/responsibilities
        \item {\textbf{Biophysical Chemistry Lab{\scriptsize (CHEM 4582)}} - 6 semesters - Instructed \textasciitilde8 undergraduate students on experimental and computational protocols.}
        \item {\textbf{Macromolecular Structure{\scriptsize (CHEM 6572)}} - 2 semesters - Directed \textasciitilde25 graduate students on the use of computational modelling programs.}
        \item {\textbf{Survey of Biochemistry{\scriptsize (CHEM 3511)}} - 1 semester - Guided \textasciitilde40 undergraduate students to solve homework problems in weekly recitation.}
      \end{cvitems}
    }

 %---------------------------------------------------------
\cventry
{Course Project for BIOL 8803b - Programming for Bioinformatics and BIOL 7210 - Computational Genomics} % Job title
{\href{http://gbrowse2015.biology.gatech.edu/Home.html}{Analysis and Interpretation of NGS Data from CDC}} % Organization
{Atlanta, GA} % Location
{Aug 2014 - May 2015} % Date(s)
{
	\begin{cvitems} % Description(s) of tasks/responsibilities
		\item Worked in multidisciplinary teams of biologists and computer scientists to identify pathogens from DNA sequences provided by the CDC.
		\item Analyzed 97 NGS single-end and paired-end reads of \textit{Neisseria meningitidis}, \textit{Haempophilus influenza}, and \textit{Haemophuilus haemolyticus} generated from GAII or Illumina HiSeq/MiSeq instruments.
		\item {\bf Results}: Developed a typing-tool that identifies the organism and its serotype/serogroup from DNA sequence file inputs and constructed a genome browser of 53 annotated genomes to view annotated genomes.
	\end{cvitems}
}

%---------------------------------------------------------
%  \cventry
%    {Michigan State University; Heinrich-Heine Universit{\"a}t} % Job title
%    {Undergraduate Research Assistant} % Organization
%    {East Lansing, MI; D{\"u}sseldorf, DE} % Location
%    {Feb 2010 - May 2012} % Date(s)
%    {
%      \begin{cvitems} % Description(s) of tasks/responsibilities
%        \item {\textbf{Project}: Molecular simulations of Mismatch Repair Enzymes MutS$\alpha$ and MutS$\beta$. Coauthored paper}
%        \item {\textbf{Project}: DNA-protein interaction of cis-regulatory elements in {\it Flaveria sp.}}
%        \item {\textbf{Project}: Metabolic flux analysis of carbon in {\it Nanochloropsis sp.}}
%        \item {\textbf{Project}: Aquaporin signalling in {\it A. thaliana} gametogensis}
%        \item {\textbf{Project}: Protein-protein interactions in ER to chloroplast lipid trafficking. {\it E. coli} and {\it S. cerevisiae}}
%      \end{cvitems}
%    }

%\vspace{-4.0mm}
%---------------------------------------------------------    
\vspace{-4.0mm}
\end{cventries}
