%!TEX TS-program = xelatex
%!TEX encoding = UTF-8 Unicode
%%%%%%%%%%%%%%%%%%%%%%%%%%%%%%%%%%%%%%%%%
% Twenty Seconds Resume/CV
% LaTeX Template
% Version 1.0 (14/7/16)
%
% Original author:
% Carmine Spagnuolo (cspagnuolo@unisa.it) with major modifications by 
% Vel (vel@LaTeXTemplates.com) and Harsh (harsh.gadgil@gmail.com)
%
% License:
% The MIT License (see included LICENSE file)
%
%%%%%%%%%%%%%%%%%%%%%%%%%%%%%%%%%%%%%%%%%

%----------------------------------------------------------------------------------------
%	PACKAGES AND OTHER DOCUMENT CONFIGURATIONS
%----------------------------------------------------------------------------------------

\documentclass[letterpaper]{twentysecondcv} 

% Command for printing skill overview bubbles
\newcommand\skills{ 
~
	\smartdiagram[bubble diagram]{
        \textbf{Structural}\\\textbf{~~Bioinformatics~~},
        \textbf{~~~~~~~~OOP~~~~~~~~~},
        \textbf{~~Molecular~~}\\\textbf{Modeling},
        \textbf{~~~Scientific~~~}\\\textbf{Writing},
        \textbf{Data}\\\textbf{Visualization},
        \textbf{~~~~Network~~~~}\\\textbf{Science},
        \textbf{Machine}\\\textbf{~~~Learning~~~}
    }
}

% Programming skill bars
\programming{{JavaScript / 1}, {SQL / 2}, {Python\tiny{ Numpy$\textbullet$Pandas$\textbullet$SK-Learn$\textbullet$NetworkX$\textbullet$BioPython} / 3.5}}

% Projects text
\projects{
	\textbf{Ribosomal PPI Evolution:} Ribosomal protein-protein interface classification \\
	\textbf{Ribosomal Protein Folding Evolution:} Inferred protein folding evolution from the structure of ribosomal proteins \\
	\textbf{PyRo:} Python module for analyzing\\Ribosomal mmCIF files \\
    \textbf{CS 7280$\textbullet$Network Science:} Atomic\\network analysis of rRNA evolution \\
    \textbf{BIOL 7210$\textbullet$Computational Genomics:} Built a genome browser from the analysis of raw NGS reads \\
}

%----------------------------------------------------------------------------------------
%	 PERSONAL INFORMATION
%----------------------------------------------------------------------------------------
% If you don't need one or more of the below, just remove the content leaving the command, e.g. \cvnumberphone{}

\cvname{Nicholas Attila \newline Kovacs} % Your name
\cvjobtitle{Structural Bioinformaticist} % Job
% title/career

\cvlinkedin{/in/KovacsNicholas}
\cvgithub{NicholasAKovacs}
\cvnumberphone{(619) 535-8895} % Phone number
\cvsite{NicholasAKovacs.com} % Personal website
\cvmail{Nicholas.Kovacs@gatech.edu} % Email address

%----------------------------------------------------------------------------------------

\begin{document}

\makeprofile % Print the sidebar

%----------------------------------------------------------------------------------------
%	 EDUCATION
%----------------------------------------------------------------------------------------
\section{Education}

\begin{twenty} % Environment for a list with descriptions
	\twentyitem
    	{Dec 2018}
        {\textsuperscript{(Expected)}}
        {\href{http://http://bioinformatics.gatech.edu/phd-curriculum/}PhD, Bioinformatics}
        {{Georgia Institute of Technology, Atlanta, GA, USA}}
        {}
        {}
	\twentyitem
    	{May 2012}
		{}
        {BS, Biochemistry}
        {{Michigan State University, East Lansing, MI, USA}}
        {}
        {}
	%\twentyitem{<dates>}{<title>}{<organization>}{<location>}{<description>}
\end{twenty}

\section{Experience}
\begin{twenty}
	\twentyitem
    	{Aug 2013 -}
		{Dec 2018}
        {\href{https://ww2.chemistry.gatech.edu/~lw26/}{PhD Candidate, Graduate Research Assistant}}
        {Georgia Inst. of Tech.}
        {}
        {
        \textbf{Adviser}: Dr. Loren Williams \\ 
        \textbf{Thesis}: The History of Proteins Revealed by Data Mining the Ribosome
        {\begin{itemize}
        \item \textbf{Hypothesis}: The ribosome is a molecular fossil; its strucuture can be mined to unravel the evolution of life 
        \item \textbf{Tools}: Python, PyMOL, Adobe Illustrator, JavaScript
        \item \textbf{Funding}: NASA Astrobiology Institute
        \item \textbf{Support}: Data analysis for experimental labmates
 \vspace{4mm}
		\end{itemize}}
        }

	\twentyitem
    	{Mar 2017 -}
		{Mar 2018}
        {\href{https://www.nsf.gov/funding/pgm_summ.jsp?pims_id=5284}{East Asia and Pacific Institutes Fellow}}
        {National Taiwan University}
        {}
        {
\vspace{-3mm}
		\textbf{Project}: The Evolution of Proteins in Eukaryotes
        {\begin{itemize}
        \item \textbf{Tools}: Python, PyMOL
        \item \textbf{Funding}: National Science Foundation \scriptsize{- East Asia and Pacific Summer Institutes}
        \item \normalsize{Independently wrote grant to conduct international research}
        \item Awarded stipend, living allowance, and airfare to Taipei, Taiwan
 \vspace{4mm}
		\end{itemize}}
        }

	\twentyitem
    	{Jan 2017 -}
		{Dec 2017}
        {Petit Scholar Mentor}
        {\href{http://petitinstitute.gatech.edu/become-petit-scholar-mentor}{Georgia Inst. of Tech.}}
        {}
        {
\vspace{-3mm}
        {\begin{itemize}
        \item Mentored undergraduate in laboratory project
        \item Awarded travel and materials allowance
 \vspace{4mm}
		\end{itemize}}
        }

        	\twentyitem
    	{Aug 2013 -}
		{Dec 2016}
        {PhD Candidate, Graduate Teaching Assistant}
        {\href{https://ww2.chemistry.gatech.edu/~lw26/}{Georgia Inst. of Tech.}}
        {}
        {
       	\textbf{Course}: Macromolecular Structure - 2 semesters
        {\begin{itemize}
        \item CHEM 6572, Graduate-level
        \item Trained $\sim$25 students on the use of molecular modeling software
		\end{itemize}}
       	\textbf{Course}: Biochemistry Lab II - 7 semesters
        {\begin{itemize}
        \item CHEM 4582, Undergraduate-level
        \item Instructed biophysical chemistry laboratory course of 8 students
 \vspace{2mm}
		\end{itemize}}
        }
\end{twenty}

%----------------------------------------------------------------------------------------
%	 PUBLICATIONS
%----------------------------------------------------------------------------------------

\section{Publications\scriptsize{(selected)}}\href{https://scholar.google.com/citations?user=KIFFrX4AAAAJ&hl=en}
- {\bf Kovacs, N.A.}, Petrov, A.S., Lanier, K.A., Williams, L.D. 2017 ``Frozen in Time: The History of Proteins", {\it Mol. Biol. Evol.} {\bf 34}, 1252-1260\vspace{2.0mm}\\
- Gómez Ramos, L. M., Degtyareva, N. N., {\bf Kovacs, N. A.}, Holguin, S. Y., Jiang, L., Petrov, A. S., Williams, L. D. 2017 ``Eukaryotic Ribosomal Expansion Segments as Antimicrobial Targets" {\it Biochemistry} {\bf 56}, 5288−5299\vspace{2.0mm}\\
- Petrov, A. S., Gulen, B., Norris, A. M., {\bf Kovacs, N. A.}, Bernier, C. R., Lanier, K. A., Williams, L. D. 2015 ``History of the ribosome and the origin of translation" {\it Proc. Natl. Acad. Sci. U.S.A.} {\bf 112}, 15396–15401\\

%----------------------------------------------------------------------------------------
%	 Presentations (experience seciton redefined)
%----------------------------------------------------------------------------------------

\section{Presentations\scriptsize{(selected)}}

\begin{twenty} % Environment for a list with descriptions
\twentyitem
    	{Jun 2017}
		{}
        {Astrobiology Graduate Student Conference}
        {{Charlottesville, VA}}
        {\textbf{Title}: The History of Proteins}
%\href{https://www.youtube.com/watch?v=FKvyR4HPFNI}

	\twentyitem
    	{Apr 2017}
		{}
        {Graduate Research Symposium}
        {\href{http://www.calendar.gatech.edu/event/590513}{Atlanta, GA}}
        {\textbf{Title}: Eukaryotic Ribosomal Protein Evolution \\ Awarded 3rd place}

	\twentyitem
    	{Dec 2016}
		{}
        {Search for Life: From Early Earth to Exoplanets}
        {{Quy Nhon, Vietnam}}
        {\textbf{Title}: Frozen in Time: The History of Proteins}
%\href{https://www.youtube.com/watch?v=5p0opfZxpDQ}

	%\twentyitem{<dates>}{<title>}{<location>}{<description>}
\end{twenty}

\end{document} 
